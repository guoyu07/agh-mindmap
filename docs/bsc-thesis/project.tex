%
%   Copyright 2013 Katarzyna Szawan <kat.szwn@gmail.com>
%       and Michał Rus <m@michalrus.com>
%
%   Licensed under the Apache License, Version 2.0 (the "License");
%   you may not use this file except in compliance with the License.
%   You may obtain a copy of the License at
%
%       http://www.apache.org/licenses/LICENSE-2.0
%
%   Unless required by applicable law or agreed to in writing, software
%   distributed under the License is distributed on an "AS IS" BASIS,
%   WITHOUT WARRANTIES OR CONDITIONS OF ANY KIND, either express or implied.
%   See the License for the specific language governing permissions and
%   limitations under the License.
%

\chapter{Project}
\label{chap:project}

\section{Requirements}
\label{sec:requirements}

\section{Planned solution}
\label{sec:plan}

\subsection{Data representation}
\label{subsec:data-repr}

Mind maps edited in our software will be kept in a database which consists of two tables. \todo{Entities diagram of the DB}One represents a mind map with a UUID, and the other --- a single node which has the following fields: UUID, its mind map UUID, content, parent (which keeps UUID of a parent node), \todo{timestamp \& conflict are hard to understand just from this}timestamp of the last modification (provided by server, this is not a local time) and a flag which says whether there's a conflict in the content of this node. The fact that a single node remembers only its parent may  cause a number of problems with conflicts in which one user deletes (online) a whole branch and other modifies its nodes offline. However, it also makes database store no excess data.

\subsection{Collaboration}
\label{subsec:collaboration}

The most challenging part of the \todo{can we use `thesis' as an alias for `paper'/`implementation'?}thesis is implementing online collaboration and synchronisation of a map when one of contributors lost the Internet connection.
