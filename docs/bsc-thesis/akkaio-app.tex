%
%   Copyright 2013 Katarzyna Szawan <kat.szwn@gmail.com>
%       and Michał Rus <m@michalrus.com>
%
%   Licensed under the Apache License, Version 2.0 (the "License");
%   you may not use this file except in compliance with the License.
%   You may obtain a copy of the License at
%
%       http://www.apache.org/licenses/LICENSE-2.0
%
%   Unless required by applicable law or agreed to in writing, software
%   distributed under the License is distributed on an "AS IS" BASIS,
%   WITHOUT WARRANTIES OR CONDITIONS OF ANY KIND, either express or implied.
%   See the License for the specific language governing permissions and
%   limitations under the License.
%

\section{Akka.io application}
\label{sec:akka-app}

Akka component was implemented almost entirely according to our project (with the only exception of blocking simultaneous edits to the same leaf). In order to make it possible to communicate between Android devices and actor system on the server-side, we have created one REST web service with two paths using the DSL of Spray.io. Any number of users can be working on a mind map, as each device gets its own actors: one for polling (Poller) and one for updates (Updater).

Two-directional communication is implemented by means of long-polling: a mobile app (or any other client) initiates a connection with a REST service which waits---before sending a response---until its actor receives a message from another actor \emph{or} a timeout of 20~seconds occurs. See \cref{subsec:android-akka-comm} for theoretical details.  

\subsection{Actor system}
\label{subsection:akka-actors}
The actor model is described in \cref{subsection:akka-actors}. `Since Akka enforces parental supervision, every actor is supervised and (potentially) the supervisor of its children'~\cite{AkkaDoc:2013:Actors}. In order to receive messages, each actor has to implement a \inlinecode{receive} method, which describes its initial behavior. Behaviors can be switched later (one way to create a Finite State Machine).

In our application we have a number of actors which perform various functions.

\begin{description}
	\item[Supervisor] not the real Akka's supervisor, but an user-land supervisor. The main actor that is started in \inlinecode{main()} method of the application. It starts MapsSupervisor and http.Service (described below).
	\item[MapsSupervisor] This is the parent of all MindMaps, creating them as the need arises. It provides means of getting appriopriate MindMap's ActorRef for a given map UUID. Pollers can subscribe to it, which results in this subscription message being routed to all MindMaps.
	\item[http.Service] Starts a HTTP listener (service) on a given port and address. Provides two paths: \begin{itemize}
		\item \inlinecode{POST /update} starts a new http.Updater actor,
		\item \inlinecode{GET /poll/since/\$TIME} starts a new http.Poller.
	\end{itemize}
	\item[MindMap] represents one concrete mind map on the server. It allows Pollers to subscribe to its changes (notifying them if any occur). It also processes messages from Updaters, ones that update the map in one way or another.
	\item[http.Poller] represents a long-polling connection to the server.
	\item[http.Updater] represents a \inlinecode{POST}-ed update of some mind map.
\end{description}

\subsection{Data storage and Squeryl}
\label{subsection:akka-database}

We have decided to store our data in a PostgreSQL database. It does not scale well horizontally (to a cluster of many servers), so it is probably a subject of a near change (to MongoDB, probably).

Another problem with SQL is that its query is just a primitive \inlinecode{String} of characters, unchecked at compile time. One cannot be sure whether it is correct or not until it fails at runtime. Therefore we have also decided to use Squery, a DSL for object-relational mapping in Scala for SQL, checked \emph{completely} at compile time, thus providing us with a maximum type safety of the mappings. Automatic checking of correctness at compile time turned out to be invaluable.

All tables kept their structure from the project stage, see \cref{fig:erd}.

\subsection{Spray and JSON}
\label{subsection:akka-spray}
Spray is an additional layer which enables communication over HTTP between Akka and client applications using JSON format messages (or any other format, but JSON is by far the most popular).

In our application, we had to create a trait \inlinecode{CustomJsonFormats}, which does necessary additions (type-safe UUID conversion) to the default JSON protocol provided by Spray. Extending the functionality of an existing library in Scala is used extensively and is possible thanks to so called \emph{type classes}, a construct of Scala's type system that supports ad hoc polymorphism.

\subsection{Synchronization}
\label{subsection:akka-synchro}
As previously stated, the synchronization is based \emph{only} on the server time, as planned. Client times could be unsynchronized with atomic time (and most probably would be) which renders them completely undependable (not to mention malicious users).

Each client does two things completely asynchronously.

\begin{enumerate}
	\item Polls for updates using \inlinecode{GET /poll/since/\$TIME}: \begin{enumerate}
		\item Akka responds with a JSON-encoded list of updates that happened \emph{after} this \inlinecode{\$TIME},
		\item one update looks like:\begin{verbatim}
{
  "mindMap": "9c91d4d4-fd6d-47da-a412-13ea931cd3ef",
  "node": {
    "uuid": "006392b3-797f-4a43-99f0-cf27748f54d9",
    "parent": "05acfa1c-1bc8-4b76-a200-b8ac8414c5b5",
    "ordering": 50.0,
    "content": "ale zabawa",
    "hasConflict": false,
    "cloudTime": 1389035195150
  }
}
		\end{verbatim}\ldots so it has its own \inlinecode{cloudTime} value,
		\item maximum of these cloud times (number of milliseconds since UNIX Epoch) is then saved at client-side and then used to poll again,
		\item and it is worth noting that a client \emph{will receive its own} updates while polling.
	\end{enumerate}
	\item Sends updates using \inlinecode{POST /update}: \begin{enumerate}
		\item the body of each update request is, too, a JSON object that looks like:\begin{verbatim}
{
  "mindMap": "9c91d4d4-fd6d-47da-a412-13ea931cd3ef",
  "nodes": [{
    "uuid": "006392b3-797f-4a43-99f0-cf27748f54d9",
    "parent": "05acfa1c-1bc8-4b76-a200-b8ac8414c5b5",
    "ordering": 50.0,
    "content": "ale zabawa",
    "hasConflict": false,
    "cloudTime": 0
  }, ...]
}
		\end{verbatim} \ldots so many nodes may be updated in one request but of only one map. Client has to group its pending updates by their map UUID,
		\item Akka responds with a list of UUID of so called \emph{orphan nodes}---nodes for which Akka does not know their respective parents---and if the list is empty, it means that everything worked,
		\item if, on the other hand, the list is non-empty, the client has to resend all the subtrees with roots being equal to parents of these \emph{orphan nodes}.
	\end{enumerate}
\end{enumerate}

It is, again, worth noting, that the server does not keep any state of the connected devices (i.e. the \emph{sessions}). Every device keeps track of what it knows itself and asks the server only for the resources it needs. Therefore, REST's statelessness constraint holds.
