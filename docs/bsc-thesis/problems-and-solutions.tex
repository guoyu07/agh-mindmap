%
%   Copyright 2013 Katarzyna Szawan <kat.szwn@gmail.com>
%       and Michał Rus <m@michalrus.com>
%
%   Licensed under the Apache License, Version 2.0 (the "License");
%   you may not use this file except in compliance with the License.
%   You may obtain a copy of the License at
%
%       http://www.apache.org/licenses/LICENSE-2.0
%
%   Unless required by applicable law or agreed to in writing, software
%   distributed under the License is distributed on an "AS IS" BASIS,
%   WITHOUT WARRANTIES OR CONDITIONS OF ANY KIND, either express or implied.
%   See the License for the specific language governing permissions and
%   limitations under the License.
%

\section{Encountered problems and their solutions}
\label{sec:impl-problems}

\subsection{Encapsulation of bidirectional message passing over request-response style HTTP protocol}
\label{subsec:problem-longpolling}

Most web browsers have some limit set for concurrently open connections to the same server. This means it would be best to encapsulate message passing in \emph{one} `concurrently' open connection and leave the others to be used in any way needed (e.g. to download media resources).

This is not much of an issue when Android is concerned, because as many connections can be opened, as we need. However, since it is not difficult to add other front-ends (web application), it is also wise to implement the REST actors (\cref{subsection:akka-actors}) and the client to be able to operate on one sequentially opened connection only.

What has to be done on the client-side is to disconnect/destroy the connection with a request of \inlinecode{GET /poll/since/\$TIME} and only then send \inlinecode{POST /update}. This is almost equally straightforward with Spray: we have to overload \inlinecode{onConnectionClosed} method of \inlinecode{HttpServiceBase} and in its new, overloaded body stop an appriopriate Poller actor.

The final choice whether to use one or two connections is left to the client: to our Akka backend both of these cases are of completely no difference.

\subsection{No removing of tabs in Android's TabHost}
\label{subsec:problem-tabhost}

Strange as it may seem, the system-provided TabHost component does not allow a removal of tabs on the same level of abstraction as it allows adding them. To remove a tab, one has to find the tab's \emph{raw} view and remove the view from its parent, the TabHost. On the other hand, to add a new tab, one has to create a TabSpec object and add this object to the TabHost. Quite a discrepancy.

It has been decided to keep track of all created TabSpecs manually and at removal request, remove \emph{all} of added tabs, remove a corresponding TabSpec from our collection and then recreate all tabs from this just reduced collection.

\subsection{No bidirectional scroll view in Android standard components}
\label{subsec:problem-scrollview}
In our application we implemented multi-directional scrolling in a class \inlinecode{Horizontal ScrollViewWithPropagation} which extends \inlinecode{HorizontalScrollView}. Inside, there is a \inlinecode{ScrollView} variable. If this inner \inlinecode{ScrollView} is set, we obtain o copy of \inlinecode{MotionEvent}, then we transform this copy in X axis and dispatch to the inner \inlinecode{ScrollView}. In the end, the original \inlinecode{MotionEvent} is handled in \inlinecode{Horizontal ScrollViewWithPropagation}.

\subsection{Positioning child nodes}
\label{subsec:problem-positioning}

Positioning of the nodes on a map was the thing that took us the most time. (While the most thinking, on the other hand, was taken by synchronization issues.) This is a completely custom controller. We had to create a view that knows exactly how to position its child views according to not-really-well defined rules for positioning nodes of a mind map.

Much thinking resulted in the following algorithm for repainting a mind map.

\begin{enumerate}
	\item Set the root's (main) node position to $(0,0)$,
	\item Find all children of the root node and sort them by their ordering parameter.
	\item Wrap 1st-level nodes in a SubtreeWrapper class. This class represents a size of a bounding rectangle for the wrapped node (any node but the root) and all of its children, the \emph{subtree}.
	\item Recursively calculate the sizes of all SubtreeWrappers using a kind of a DFS algorithm.
	\item, having the sizes of bounding rectangles for 1st-level subtrees, divide these subtrees into two `almost-halves' by their heights: sum of heights of the subtrees in the left half should be as close as possible to the sum of heights in the right one.
	\item Now it is trivial to set $y$ positions of the subtrees. $x$ positions of the 1st-level subtrees are set using a half of the $\sin(Cy)$ function.
	\item Having set the positions of the 1st level, it is relatively easy to set the positions of the 2nd level, and so on, recursively. This time it is a BFS-type algorithm.
	\item After we have all positions set, all that is left to do is actual drawing of the inflated \inlinecode{mind\_node.xml} views along with arrows connecting children to their parents.
	\end{enumerate}
