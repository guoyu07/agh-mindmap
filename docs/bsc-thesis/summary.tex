%
%   Copyright 2013 Katarzyna Szawan <kat.szwn@gmail.com>
%       and Michał Rus <m@michalrus.com>
%
%   Licensed under the Apache License, Version 2.0 (the "License");
%   you may not use this file except in compliance with the License.
%   You may obtain a copy of the License at
%
%       http://www.apache.org/licenses/LICENSE-2.0
%
%   Unless required by applicable law or agreed to in writing, software
%   distributed under the License is distributed on an "AS IS" BASIS,
%   WITHOUT WARRANTIES OR CONDITIONS OF ANY KIND, either express or implied.
%   See the License for the specific language governing permissions and
%   limitations under the License.
%

\chapter{Summary}
\label{chap:summary}

\section{Testing the applications}
\label{sec:summary-testing}
During the final stage of creating our application, we had to find a way to verify how it met the requirements and whether all the functionalities we implemented work in an anticipated way. We conducted application tests, mostly by ourselves. We also asked two of our friends to test this application for us. We wanted to find any bugs and verify whether the user interface is intuitive and handy. The results were satisfying - the users did not have problems using the application. The designed synchronization algorithm works as we thought. Also, the technologies we used for writing the server part turned out to be robust.

\section{How does the implementation solve the problem?}
\label{sec:summary-how-solve}
The aims of our work are put in \cref{sec:requirements}. Now we will sum up what we managed to implement. 
 
The first objective was to create an application compatible with XMind which lets its users create, edit and display mind maps. We succeeded in this task: our application enables creating, editing and displaying mind maps. We did not implement deleting mind maps, but a user can delete a node with its subtree,  apart from root node.

Next, there was a part about presenting maps in a clean, structured way. We believe our application meets tis requirement. We put a lot of effort in positioning mind nodes, bidirectional scroll view also turned out to be troublesome and it is described in the \cref{sec:impl-problems}.

The collaboration works very well, just as it is described in the requirements, with one exception, mentioned in the next section. The algorithm of synchronization we designes in the project part worked  as we anticipated. 

\section{What is missing in our solution?}
\label{sec:summary-missing}
The Android application is visually compatibile with XMind only on a very basic level. The main reason for that is that that was not the main issue while creating the software. We wanted to focus on creating the \emph{collaboration} tool. Still, a better compatibility would definitely be beneficial. 

When it comes to XMind functionalities, there are two which are quite useful. First is the ability to mark a relation between two nodes. We did not implement this feature due to its UI complexity. Adding it on a backend level (storage in database and importing from XMind files) is simple, but at this point our system of positioning supports tree structure. A line connecting any two nodes should  bypass other nodes. The second is adding longer notes to a single node. That would also be pretty easy to implement on data model level, but again the problem is UI: we did not came across a suitable way of displaying, editing or deleting notes.

When it comes to the UI, we did not implement deleting whole mind maps, which seems a basic functionality in mind mapping software. However, we had a conceptual problem with how to solve this in collaboration. There was a problem with how to restore a mind map if one user deletes it and other would change its content.

 Subtree folding also seems very important (althought not mentioned in our requirements). 
 
 Another missing feature is moving any node (reassigning its parent node) by long-pressing it and choosing `Change parent' option, and then tapping on the new parent-to-be was not implemented. At that point, there was not enough time left to add it.

Speaking of collaboration, there is one thing we did not implement: blocking edited leafs. It is an important for application usability, but due to the limited time we had, we did not manage to work our an appropriate solution. 

\section{Future works}
\label{sec:summary-future}
When it comes to what could be done next, some of the things we would like to see implemented include:

\begin{itemize}
	\item adding users, privileges, explicit sharing etc. (the most important feature the system is now lacking),
	\item better XMind compatibility, support for node--node relations, folding, floating objects and attachments, custom icons, as well as some predefined styles,
	\item a web application using only static HTML/JS/CSS files (and, what's most important, \emph{the same} highly-scalable Akka as a backend),
	\item a move of Akka's storage from PostgreSQL to a more scalable NoSQL store (probably MongoDB),
	\item an involvement of a professional UI designer; having him draw all UI elements would add to a beautiful user experience,
	\item making use of a new \inlinecode{akka-cluster} Akka subproject to have no central/main Akka node; instead, a completely self-organizing cluster of servers would be more fault-tolerant,
	\item adding map-wide blocking of currently-edited node for other, non-editing users to improve the UX even more.
\end{itemize}
