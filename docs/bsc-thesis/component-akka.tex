%
%   Copyright 2013 Katarzyna Szawan <kat.szwn@gmail.com>
%       and Michał Rus <m@michalrus.com>
%
%   Licensed under the Apache License, Version 2.0 (the "License");
%   you may not use this file except in compliance with the License.
%   You may obtain a copy of the License at
%
%       http://www.apache.org/licenses/LICENSE-2.0
%
%   Unless required by applicable law or agreed to in writing, software
%   distributed under the License is distributed on an "AS IS" BASIS,
%   WITHOUT WARRANTIES OR CONDITIONS OF ANY KIND, either express or implied.
%   See the License for the specific language governing permissions and
%   limitations under the License.
%

\subsection{Component: Akka.io application}
\label{subsec:component-akka}

\todo[inline]{\michal{M., project Akka: what subcomponents? (Own and from library!)}}

Akka is used as a backend for the mobile application. It enables all mobile devices with the application installed to share maps and collaborate on them in real-time. Several REST web services are implemented using Spray.io (a REST interface to Akka) for communication between Android devices and actor system on the server-side.

Each REST-connected mobile device gets its own actor. Instant bidirectional communication between devices is achieved by means of `long-polling': mobile app initiates a connection with a REST service which does not respond until its actor receives a message from another actor.

See \cref{subsec:android-akka-comm} for details about Android--Akka communication protocol.

\todo[inline]{\michal{M., what messages do the actors \emph{within} the system exchange?}}

\todo[inline]{\michal{M., how does Akka communicate with Postgres?}}
