%
%   Copyright 2013 Katarzyna Szawan <kat.szwn@gmail.com>
%       and Michał Rus <m@michalrus.com>
%
%   Licensed under the Apache License, Version 2.0 (the "License");
%   you may not use this file except in compliance with the License.
%   You may obtain a copy of the License at
%
%       http://www.apache.org/licenses/LICENSE-2.0
%
%   Unless required by applicable law or agreed to in writing, software
%   distributed under the License is distributed on an "AS IS" BASIS,
%   WITHOUT WARRANTIES OR CONDITIONS OF ANY KIND, either express or implied.
%   See the License for the specific language governing permissions and
%   limitations under the License.
%
\subsection{REST}
\label{subsec:restful}
 Communication between Android devices and actor system on the server-side requires a REST web service with two paths, created using the DSL of Spray.io. It enables communication over HTTP between Akka and client applications using JSON-format messages. We will be using GET and POST HTTP methods in order to request and send updates. Default JSON (un-)marshallers provided by Spray.io are not sufficient for sending our application's data. We will also need a custom (un-)marshaller to handle the UUID type.

\subsection{Communication}
\label{subsec:android-akka-comm}

Real-time collaboration (along with \emph{offline} synchronization) is by any means the most challenging part of designing and later implementing the system.

It needs mentioning that our solution is not really based on any of the existing ones. It is being created completely and from scratch by us. Such a highly parallelized system is hard to formally reason about and is yet to be proved (However, counting on a \emph{very strict} Scala compiler along with proper types and abstractions, later turned out to be a sufficient `kind-of-proof,' as the implementation works correctly. This is often a case with languages which favor immutability and functional programming: if it compiles, then it is correct). The solution will be verified in testing the application. 

As it turns out, we are able to make it work using just two main messages (and resolving two exceptional cases). However, the fact that a single node remembers only its parent will cause a number of problems with merge conflicts dealing with node deletions (solved by subtree recreation algorithm in \cref{subsec:subtree-recreation}). Communication with an Android device from the \emph{moment it gets online} is described below.

\begin{enumerate}
	\item Android looks for the newest timestamp in \inlinecode{mind\_node} table. This is the last time we had contact with Akka.

	\item Android sends a message to Akka. The message contains:
	\begin{itemize}
		\item request for all changes that happened after that \emph{newest timestamp},
		\item all local changes, that is a list of all nodes that have their timestamp set to \inlinecode{NULL}. This list of modified nodes needs to be sorted so that \emph{parents} come before \emph{children} for any two nodes in the list. \Cref{akka-unknown-parent} of exceptional cases below explains why.
	\end{itemize}

	\item Akka receives the message and:
	\begin{itemize}
		\item creates mind map(s) of UUID contained in node's map UUID field (if they don't already exists),
		\item checks if the DB will be in consistent state after introducing Android's changes. DB is said to be inconsistent when there are references in node's parent field to other nodes that are not present in DB (dangling references),
		\item any deletions are rejected if there is a previously synchronized change in to-be-deleted subtree (in such case this to-be-deleted subtree is sent again to the deleter in message in \cref{akka-update-message}),
		\item changes are merged if consistency will be retained afterwards, node's timestamps are set to current server time,
		\item if, however, consistency cannot be retained, Akka behaves in accordance to \cref{akka-unknown-parent} of exceptional cases below.
		\item There's also a possibility of a conflict, see \cref{akka-conflict} of exceptional cases below.
	\end{itemize}

	\item \label{akka-update-message} After any change to server-side DB, Akka notifies all connected Androids, including the one that initiated the change. The notification message contains all changed nodes sorted in aforementioned parent-first order.

	\item Android, after receiving this update message, updates its local DB.
\end{enumerate}

These aforementioned two exceptional cases are described below.

\begin{enumerate}
	\item \label{akka-conflict} It might happen that another Android, A, updated a node, when currently synchonizing Android, B, was offline (and B, too, changed the node while offline). This causes a `merge conflict'. All merge conflicts are resolved automatically and strategies vary depeneding on which field causes the conflict:
	\begin{itemize}
		\item if there is a conflict in node's \emph{content}, then both conflicting contents are concatenated with a new line, \inlinecode{\textbackslash{}m} and the node's \inlinecode{has\_conflict} field is set to \inlinecode{true} (this results in marking the field somehow in the UI, possibly with red),
		\item if, however, the conflicting field is node's \inlinecode{parent}, then we \emph{repin} the node accordnig to our last disposition. Changing node's parent does not destroy data and it is easy to reverse for contributors.
	\end{itemize}

	\item \label{akka-unknown-parent} It is possible that Akka will get an update request with a node which parent is \emph{not} in the DB. This might happen for two reasons:
	\begin{itemize}
		\item malicious request,
		\item or, more probably, the request is trying to update a subtree that was previously deleted (and this deletion was not synchronized at the moment of `physical' update).
	\end{itemize}

	List of updated nodes send in the request is sorted in parent-first order. This allows for a performance boost at receiving site. Akka needs only to look at modified node list's head to check whether it knows about parents of updated nodes.

	In the second case, subtree recreation algorithm has to be used (\cref{subsec:subtree-recreation}).
\end{enumerate}

