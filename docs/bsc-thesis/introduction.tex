%
%   Copyright 2013 Katarzyna Szawan <kat.szwn@gmail.com>
%       and Michał Rus <m@michalrus.com>
%
%   Licensed under the Apache License, Version 2.0 (the "License");
%   you may not use this file except in compliance with the License.
%   You may obtain a copy of the License at
%
%       http://www.apache.org/licenses/LICENSE-2.0
%
%   Unless required by applicable law or agreed to in writing, software
%   distributed under the License is distributed on an "AS IS" BASIS,
%   WITHOUT WARRANTIES OR CONDITIONS OF ANY KIND, either express or implied.
%   See the License for the specific language governing permissions and
%   limitations under the License.
%

\chapter{Introduction}
\label{chap:introduction}

% What? The problem

Mind maps are diagrams which visually gather ideas and put them in a logical structure. The structure usually consists of one or more central words, around which associated ideas and concepts are placed. Subbranches may also represent categories. They are used in a whole range of disciplines, both in science and arts, for both commercial and personal use~\cite{Gee:2010:Roots}.

% How? Shortly about the method

The subject of the thesis is project and implementation of a cloud-based mind-mapping tool for Android. It should support bidirectional data exchange with one of the most popular mind-mapping software: XMind. Also, it should provide tools for collaboration: working on the same map in real-time by more than one person. The task will be divided into two modules. First one is creating an Android application, which supports XMind files. Second part of the task, the most important, is creation of collaboration tool for many users. 

% Why? The source

Mind-mapping tools are becoming more and more popular these days. At this point, there is a shortage of an Android mind-mapping software, while the number of devices operating on Android is growing rapidly. At this point, as far as we know there is no Android application which makes it possible to work on the same map collaboratively.

% What for? Consequences

Mind-mapping tools are used in a whole range of disciplines, both in science and arts. They are proved to be a very effective way of organizing ideas, brainstorming and dealing with information overload. Considering the fact that tablets and mobile phones are nowadays a natural way of recording notes and ideas for most people, creating an application which makes the process of collecting and organizing ideas intuitive, effective and most important, makes it possible to collaborate, seems a significant task.

% What's in next chapters?

Next chapters include the theoretical analysis of the problem and detailed description of the possible solution. Then, we will discuss the implementation details. In the final part we will focus on the analysis of our solution, providing tests to check how many of the initial problems were solved by the application.

The work was split between two persons. The details of the division of tasks can be traced below in \cref{tab:who-did-impl}, \cref{tab:who-did-impl-akk} and \cref{tab:who-did-docs}.

It has been decided to share both the code and this paper under the terms of of free Apache License, version 2. This license is compatible with GNU General Public License. It is also a \emph{permissive} license, although all contributors are required to submit a patent license on contributions infringing their own patents.

\begin{table}[h]
	\centering
	\begin{tabular}{l|c|c}
		What? & K. Szawan & M. Rus \\
		\hline
		Setting up the project and adding sbt plugins &  & M \\
		Action bar & & M\\
		Maps list & & M\\
		Navigation bar with tabs & K & \\
		Basic models & K & M\\
		Opening a map in a new tab & K & M\\
		Adding a new map & K & \\
		Closing tabs & & M\\
		XMind Importer & K & \\
		Adding Importer to ActionBar & K & \\
	    Bidirectional scrolling & K & M \\
	    Dp to px conversion & K & \\
	    Map drawing helper methods  & K & \\
	    Drawing root node & K & \\
	    Drawing 1st level children on ellipse & K & \\
	    Node's division to left and right tree & K & \\
	    Positioning child nodes & & M \\
	    The final version of drawing & & M \\
	    Adding child node & K & \\
	    Deleting a node with subtree &  & M \\
	    Editing node's content &  K & \\
	    Adding database &  & M \\
	    Making operations on mind nodes save in database & K & M \\
	\end{tabular}
	\caption{Division of Android related implementation tasks.}
	\label{tab:who-did-impl}
\end{table}



\begin{table}[h]
	\centering
	\begin{tabular}{l|c|c}
		What? & K. Szawan & M. Rus \\
		\hline
		Initializing Akka project and settings & & M\\
		Adding HTTP service &  & M\\
		MindMap actor & & M\\
		Supervising actor & K & M\\
		Poller actor &  & M\\
		Service actor &  & M\\
		Updater actor &  & M\\
		Managing subscriptions &  & M\\
		Refresher and Synchronizer &  & M\\
		CustomJsonFormats trait & K & M\\
		Integrating spray-can &  & M\\
		Adding spray-json dependency &  & M\\
		Cooperation use cases & K & M\\ 
		Mind maps memoization &  & M\\
		Conflict resolution &  & M\\
		Setting up akka application on Amazon EC2 &  & M\\
	\end{tabular}
	\caption{Division of Akka related implementation tasks.}
	\label{tab:who-did-impl-akk}
\end{table}

\begin{table}[h]
	\centering
	\begin{tabular}{l|c|c}
		What? & K. Szawan & M. Rus \\
		\hline
		\Cref{chap:introduction} & K & \\
		\Cref{sec:xmind} & K & \\
		\Cref{sec:android-theory} & K & \\
		\Cref{sec:scala} &  & M \\
		\Cref{sec:akka} &  & M \\
		\Cref{sec:requirements} & K  &  \\
		\Cref{sec:plan}  & K &  \\
		\Cref{subsec:component-android} & K  & M \\
		\Cref{subsec:data-repr}  &   & M \\
		\Cref{subsec:xmind-exchange}  & K  &  \\
		\Cref{subsec:android-akka-comm}  &   & M \\
		\Cref{subsec:subtree-recreation}  &   & M \\		
		\Cref{subsec:choosing-ide} & K  &  \\
		\Cref{subsec:good-idea} & K  &  \\
		\Cref{subsec:android-sbt} & K  &  \\
		\Cref{subsec:drawing} & K  &  \\
		\Cref{subsec:drawing-m} & K  &  \\
		\Cref{subsec:creating-mm} & K  &  \\
		\Cref{subsec:import} & K  &  \\
		\Cref{subsection:akka-actors} &   & M \\
		\Cref{subsection:akka-database} &  & M \\
		\Cref{subsection:akka-spray} &  & M \\
		\Cref{subsection:akka-synchro} &  & M \\
		\Cref{subsec:problem-longpolling} &  & M \\
		\Cref{subsec:problem-tabhost} &  & M \\
		\Cref{subsec:problem-scrollview} & K &  \\
		\Cref{subsec:problem-positioning} &  & M \\
		\Cref{sec:summary-testing} & K  & M \\
		\Cref{sec:summary-how-solve} & K  & M \\
		\Cref{sec:summary-missing} & K  & M \\
		\Cref{sec:summary-future} & K  & M \\
		
	\end{tabular}
	\caption{Division of thesis-related tasks.}
	\label{tab:who-did-docs}
\end{table}
