\section{Encountered problems and their solutions}
\label{sec:impl-problems}

\subsection{Encapsulation of bidirectional message passing over request-response style HTTP protocol}
\label{subsec:problem-longpolling}

\todo[inline]{\michal{M., write about long-polling.}}

Most browsers have a limit for concurrently open connections to the same server set to a value close to 2. This means it would be best to encapsulate message passing in \emph{one} constantly open connection, to leave the other free to be used in any way needed. E.g. to download some media resources.

This is not such an issue when Android is concerned, because as many connections can be opened as needed. However, as it is internally easy to add other front-ends (web application), it's wise to project the REST actors (\cref{subsec:component-akka}) to operate on one connection only.

\subsection{No removing of tabs in Android's TabHost}
\label{subsec:problem-tabhost}

\todo[inline]{\michal{M., describe custom removing of tabs.}}

Resolved: \href{https://github.com/michalrus/agh-mindmap/commit/ebe22968dc091f575ab16be0d0051dbfb7f3e434}{ebe22968dc091f575ab16be0d0051dbfb7f3e434}.

All tabs had to be removed and then readded. Also we had to manually use another container for our Fragments.

\subsection{No bidirectional scroll view in Android standard components}
\label{subsec:problem-scrollview}
In our application we implemented multi-directional scrolling in a class \inlinecode{Horizontal ScrollViewWithPropagation} which extends \inlinecode{HorizontalScrollView}. Inside, there is a \inlinecode{ScrollView} variable. If this inner \inlinecode{ScrollView} is set, we obtain o copy of \inlinecode{MotionEvent}, then we transform this copy in X axis and dispatch to the inner \inlinecode{ScrollView}. In the end, the original \inlinecode{MotionEvent} is handled in \inlinecode{Horizontal ScrollViewWithPropagation}.

\subsection{Positioning child nodes}
\label{subsec:problem-positioning}