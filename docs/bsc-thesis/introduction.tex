%
%   Copyright 2013 Katarzyna Szawan <kat.szwn@gmail.com>
%       and Michał Rus <m@michalrus.com>
%
%   Licensed under the Apache License, Version 2.0 (the "License");
%   you may not use this file except in compliance with the License.
%   You may obtain a copy of the License at
%
%       http://www.apache.org/licenses/LICENSE-2.0
%
%   Unless required by applicable law or agreed to in writing, software
%   distributed under the License is distributed on an "AS IS" BASIS,
%   WITHOUT WARRANTIES OR CONDITIONS OF ANY KIND, either express or implied.
%   See the License for the specific language governing permissions and
%   limitations under the License.
%

\chapter{Introduction}
\label{chap:introduction}

\section{\todo{Remove headlines (after having written the introduction -- it must contain all 5 modules).}What? The problem}
\label{sec:what}

Mind maps are diagrams which visually gather ideas and put them in a logical structure. The structure usually consists of one or more central words, around which associated ideas and concepts are placed. Subbranches may also represent categories. They are used in a whole range of disciplines, both in science and arts, for both commercial and personal use. \todo{This cite goes here?}\cite{Gee:2010:Roots}

\section{How? Shortly about the method}
\label{sec:how}

The subject of the thesis is project and implementation of a cloud-based mind-mapping tool for Android. It should support bidirectional data exchange with one of the most popular mind-mapping software: XMind. Also, it should provide tools for collaboration -- working on the same map in real-time by more than one person. The task will be divided into two modules. First one is creating an Android application, which supports XMind files. It will be implemented using Scala language and Android SDK. Second part of the task is creation of collaboration tool for many users. To achieve this, we are going to use Akka.io and Spray.io -- actor system with REST communication model designed for Scala. Collaboration will be implemented with a limitation: one leaf can only be edited by one person at a time. After some time of idleness, editing should be disabled to unlock the leaf for other collaborators.

\section{Why? The source}
\label{sec:why}

Mind-mapping tools are becoming more and more popular these days. At this point, there is a shortage of an Android mind-mapping software, while the number of devices operating on Android is growing rapidly.

\section{What for? Consequences}
\label{sec:whatfor}

Mind-mapping tools are used in a whole range of disciplines, both in science and arts. They are proved to be a very effective way of organising ideas, brainstorming and dealing with information overload. Considering the fact that tablets and mobile phones are nowadays a natural way of recording notes and ideas for most people, creating an application which makes the process of collecting and organising ideas intuitive and effective seems an important task.

\section{What's in next chapters?}
\label{sec:nextchapters}

Next chapters include the theoretical analysis of the problem and detailed description of the possible solution. Then, we will discuss the implementation details. In the final part we will focus on the analysis of our solution, providing tests to check how many of the initial problems were solved by the application.
