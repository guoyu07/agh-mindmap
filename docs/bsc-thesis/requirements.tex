%
%   Copyright 2013 Katarzyna Szawan <kat.szwn@gmail.com>
%       and Michał Rus <m@michalrus.com>
%
%   Licensed under the Apache License, Version 2.0 (the "License");
%   you may not use this file except in compliance with the License.
%   You may obtain a copy of the License at
%
%       http://www.apache.org/licenses/LICENSE-2.0
%
%   Unless required by applicable law or agreed to in writing, software
%   distributed under the License is distributed on an "AS IS" BASIS,
%   WITHOUT WARRANTIES OR CONDITIONS OF ANY KIND, either express or implied.
%   See the License for the specific language governing permissions and
%   limitations under the License.
%

\section{Requirements}
\label{sec:requirements}

Our aim is to create an application compatible with XMind which lets its users create, edit, delete and display mind maps. It shoud write data in its own, native format, but also implement importing and exporting maps to XMind format. It should work on most modern Android devices, both smartphones and tablets. When it comes to functionalities, the main features are connected with managing mind maps. The most important feature (apart from these, which make it possible to use our Android application) is online and offline collaboration. 

Mind maps should be presented in a clean, structured way. A special algorithm is needed to fulfill this task, because the number and size of child nodes will vary. Nodes need to be positioned around the root node. Every node can have subnodes, which are displayed as a list, and it should be possible to hide or show node's whole subtree by clicking on a special button. Adding a new node can be implemented by placing a plus button on the bottom of parent node. The whole node's body should be clickable and thus a user could enter edition mode.

It should be possible to share the same map between more than one person. Every participant should be able to edit a map (online, but also when the connection is not available) and see others' changes as soon as it is possible. This feature will probably generate a number of problems with data synchronization. When f.ex. user A deletes a whole subtree online, and other, user B, being offline, changes something in the structure of the same subtree, the application should be able to recreate the deleted structure and save it as soon as user B's connection comes back. Also, it should mark any conflicts in the content of single, atomic node. 

As this is a \todo{\igor{"zły powód braku użytkowników; dobrym powodem może być to, że np. mamy inne funkcjonalności"}}proof-of-concept system, no user privileges will be implemented (although these would be easy to add, considering highly modular and clean organization of code). Thus, any device will be able to modify any existing map.
