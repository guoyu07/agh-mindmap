%
%   Copyright 2013 Katarzyna Szawan <kat.szwn@gmail.com>
%       and Michał Rus <m@michalrus.com>
%
%   Licensed under the Apache License, Version 2.0 (the "License");
%   you may not use this file except in compliance with the License.
%   You may obtain a copy of the License at
%
%       http://www.apache.org/licenses/LICENSE-2.0
%
%   Unless required by applicable law or agreed to in writing, software
%   distributed under the License is distributed on an "AS IS" BASIS,
%   WITHOUT WARRANTIES OR CONDITIONS OF ANY KIND, either express or implied.
%   See the License for the specific language governing permissions and
%   limitations under the License.
%

\chapter{Implementation}
\label{chap:implementation}

\section{What was implemented and how?}
\label{sec:got-implemented}

\subsection{\todo{\michal{K., shouldn't this be in \cref{chap:results}?}}UI screenshots}
\label{subsec:ui-screenshots}

\todo[inline]{\michal{K., \textbackslash{}cref screenshots and their corresponding mockups in text here.}}

\begin{figure}[h]
	\centering
	\missingfigure{\michal{List view screen}}
	\caption{View of mind map list, initial screen.}
	\label{fig:screen-maplist}
\end{figure}

\begin{figure}[h]
	\centering
	\missingfigure{\michal{Mind map view screen}}
	\caption{View of mind map.}
	\label{fig:screen-map}
\end{figure}

\begin{figure}[h]
	\centering
	\missingfigure{\michal{2 devices, collaboration}}
	\caption{View of collaboration.}
	\label{fig:screen-collaboration}
\end{figure}

\section{How the requirements were met?}
\label{sec:met-requirements}

\todo[inline]{\michal{K., what goes here? Also: it doubles with 3rd to-do in \cref{chap:summary}.}}

\section{Encountered problems and their solutions}
\label{sec:impl-problems}

\subsection{Encapsulation of bidirectional message passing over request-response style HTTP protocol}
\label{subsec:problem-longpolling}

\todo[inline]{\michal{M., write about long-polling.}}

Most browsers have a limit for concurrently open connections to the same server set to a value close to 2. This means it would be best to encapsulate message passing in \emph{one} constantly open connection, to leave the other free to be used in any way needed. E.g. to download some media resources.

This is not such an issue when Android is concerned, because as many connections can be opened as needed. However, as it is internally easy to add other front-ends (web application), it's wise to project the REST actors (\cref{subsec:component-akka}) to operate one one connection only.

\subsection{No removing of tabs in Android's TabHost}
\label{subsec:problem-tabhost}

\todo[inline]{\michal{M., describe custom removing of tabs.}}

Resolved: \href{https://github.com/michalrus/agh-mindmap/commit/ebe22968dc091f575ab16be0d0051dbfb7f3e434}{ebe22968dc091f575ab16be0d0051dbfb7f3e434}.

All tabs had to be removed and then readded. Also we had to manually use another container for our Fragments.

\subsection{No bidirectional scroll view in Android standard components}
\label{subsec:problem-scrollview}

\todo[inline]{\michal{M., write about custom bidirectional ScrollView.}}
