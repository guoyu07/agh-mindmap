\section{Akka.io application}
\label{sec:akka-app}
Akka component was implemented almost entirely according to our project. In order to make it possible to communicate between Android devices and actor system on the server-side there are several REST web services from Spray.io. When more than one user is working on the mind map, each device gets its own actor.  Two-directional communication is implemented by long-polling: a mobile app initiates a connection with a REST service which waits with responding until its actor receives a message from another actor. See \cref{subsec:android-akka-comm} for theoretical details.  

\subsection{Actors system }
\label{subsection:akka-actors}

\subsubsection{Supervisor actor}
\label{subsubsection:akka-actors-supervisor}
Main supervisor actor is provided by Akka; tightly integrated with the actor system. There is always one such actor for each system.
	
\subsubsection{Per-user actors}
\label{subsubsection:akka-actors-peruser}

\subsection{Database and Squeryl}
\label{subsection:akka-database}

\subsection{Spray and JSON}
\label{subsection:akka-spray}

\subsection{Synchronization}
\label{subsection:akka-synchro}

\begin{figure}[h]
	\centering
	\missingfigure{\michal{2 devices, collaboration}}
	\caption{View of collaboration.}
	\label{fig:screen-collaboration}
\end{figure}
